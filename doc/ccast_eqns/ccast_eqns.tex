\documentclass[11pt]{beamer}
% \usetheme{Boadilla}
  \usetheme{default}


% acronyms for text or math mode
\newcommand {\ccast} {\mbox{\small CCAST}}
\newcommand {\cris} {\mbox{\small CrIS}}

\newcommand {\airs} {\mbox{\small AIRS}}
\newcommand {\iasi} {\mbox{\small IASI}}
\newcommand {\idps} {\mbox{\small IDPS}}
\newcommand {\nasa} {\mbox{\small NASA}}
\newcommand {\noaa} {\mbox{\small NOAA}}
\newcommand {\nstar} {\mbox{\small STAR}}
\newcommand {\umbc} {\mbox{\small UMBC}}
\newcommand {\uw}   {\mbox{\small UW}}

\newcommand {\fft}  {\mbox{\small FFT}}
\newcommand {\ifft} {\mbox{\small IFFT}}
\newcommand {\fir}  {\mbox{\small FIR}}
\newcommand {\fov}  {\mbox{\small FOV}}
\newcommand {\for}  {\mbox{\small FOR}}
\newcommand {\ict}  {\mbox{\small ICT}}
\newcommand {\ils}  {\mbox{\small ILS}}
\newcommand {\igm}  {\mbox{\small IGM}}
\newcommand {\opd}  {\mbox{\small OPD}}
\newcommand {\rms}  {\mbox{\small RMS}}
\newcommand {\zpd}  {\mbox{\small ZPD}}
\newcommand {\ppm}  {\mbox{\small PPM}}
\newcommand {\srf}  {\mbox{\small SRF}}

\newcommand {\ES} {\mbox{\small ES}}
\newcommand {\SP} {\mbox{\small SP}}
\newcommand {\IT} {\mbox{\small IT}}
\newcommand {\SA} {\mbox{\small SA}}

\newcommand {\ET} {\mbox{\small ET}}
\newcommand {\FT} {\mbox{\small FT}}

% abbreviations, mainly for math mode
\newcommand {\real} {\mbox{real}}
\newcommand {\imag} {\mbox{imag}}
\newcommand {\atan} {\mbox{atan}}
\newcommand {\obs}  {\mbox{obs}}
\newcommand {\calc} {\mbox{calc}}
\newcommand {\sinc} {\mbox{sinc}}
\newcommand {\psinc} {\mbox{psinc}}
\newcommand {\std} {\mbox{std}}

% symbols, for math mode only
\newcommand {\wnum} {\mbox{cm$^{-1}$}}
\newcommand {\lmax} {L_{\mbox{\tiny max}}}
\newcommand {\vmax} {V_{\mbox{\tiny max}}}

\newcommand {\tauobs} {\tau_{\mbox{\tiny obs}}}
\newcommand {\taucal} {\tau_{\mbox{\tiny calc}}}
\newcommand {\Vdc}  {V_{\mbox{\tiny DC}}}

\newcommand {\rIT} {r_{\mbox{\tiny\textsc{ict}}}}
\newcommand {\rES} {r_{\mbox{\tiny\textsc{es}}}}
\newcommand {\robs} {r_{\mbox{\tiny obs}}}

\newcommand {\rITobs} {r_{\mbox{\tiny\textsc{ict}}}^{\mbox{\tiny obs}}}
\newcommand {\rITcal} {r_{\mbox{\tiny\textsc{ict}}}^{\mbox{\tiny cal}}}

\newcommand {\ITmean} {\langle\mbox{\small IT}\rangle}
\newcommand {\SPmean} {\langle\mbox{\small SP}\rangle}


\title{ccast calibration equations}
\author{H.~E.~Motteler and L.~L.~Strow}
\institute{
  UMBC Atmospheric Spectroscopy Lab \\
  Joint Center for Earth Systems Technology \\
}
\date{\today}
\begin{document}

%----------- slide --------------------------------------------------%
\begin{frame}[plain]
\titlepage
\end{frame}
%----------- slide --------------------------------------------------%
\begin{frame}
\frametitle{calibration equation}

The \ccast\ reference calibration equation is

\[r_{\mbox{\tiny OBS}} = F \cdot r_{\mbox{\tiny ICT}}\cdot f \cdot
  \SA^{-1}\cdot f \cdot \frac{\ES - \SP}{\IT - \SP} \]

\begin{itemize}
  \item $r_{\mbox{\tiny OBS}}$ is calibrated radiance at the user grid
  \item $F$ is Fourier interpolation from sensor to user grid
  \item $f$ is a raised-cosine bandpass filter
  \item $r_{\mbox{\tiny ICT}}$ is expected ICT radiance at the sensor grid
  \item $\SA^{-1}$ is the inverse of the ILS matrix
  \item $\ES$ is earth-scene count spectra
  \item $\IT$ is calibration target count spectra
  \item $\SP$ is space-look count spectra
\end{itemize}

\end{frame}
%----------- slide --------------------------------------------------%
\begin{frame}
\frametitle{notes}

\begin{itemize}
  \item the $\IT$ and $\SP$ looks are averaged over several scans
  \item we divide the count spectra by the numeric filter
    at the sensor grid, but this cancels out in the ratio $(\ES - \SP) /
    (\IT - \SP)$
  \item $F$ is a zero-filled double Fourier interpolation
  \item $f \cdot \SA^{-1} \cdot f$ can be considered as a
    physically-based smoothing of the rows and columns of $\SA^{-1}$
\end{itemize}

\end{frame}
%----------- slide --------------------------------------------------%
\begin{frame}
\frametitle{alternate calibration equation 2}

Alternate calibration equation c2 is

\[r_{\mbox{\tiny OBS}} = F \cdot r_{\mbox{\tiny ICT}}\cdot f \cdot
   \frac{\SA^{-1}N^{-1}(\ES - \SP)}{\SA^{-1}N^{-1}(\IT - \SP)} \]

\begin{itemize}
  \item $r_{\mbox{\tiny OBS}}$ is calibrated radiance at the user grid
  \item $F$ is Fourier interpolation from sensor to user grid
  \item $f$ is a raised-cosine bandpass filter
  \item $r_{\mbox{\tiny ICT}}$ is calculated ICT radiance at the sensor grid
  \item $\SA^{-1}$ is the inverse of the ILS matrix
  \item $N^{-1}$ is the inverse of the numeric filter
  \item $\ES$ is earth-scene count spectra
  \item $\IT$ is calibration target count spectra
  \item $\SP$ is space-look count spectra
\end{itemize}

\end{frame}
%----------- slide --------------------------------------------------%
\begin{frame}
\frametitle{alternate calibration equation 1}

Alternate calibration equation c1 is

\[r_{\mbox{\tiny OBS}} = F \cdot r_{\mbox{\tiny ICT}}\cdot f \cdot
   \frac{\SA^{-1}\cdot f \cdot N^{-1}(\ES - \SP)}{\SA^{-1}\cdot f \cdot N^{-1}(\IT - \SP)} \]

\begin{itemize}
  \item $r_{\mbox{\tiny OBS}}$ is calibrated radiance at the user grid
  \item $F$ is Fourier interpolation from sensor to user grid
  \item $f$ is a raised-cosine bandpass filter
  \item $r_{\mbox{\tiny ICT}}$ is expected ICT radiance at the sensor grid
  \item $\SA^{-1}$ is the inverse of the ILS matrix
  \item $N^{-1}$ is the inverse of the numeric filter
  \item $\ES$ is earth-scene count spectra
  \item $\IT$ is calibration target count spectra
  \item $\SP$ is space-look count spectra
\end{itemize}

\end{frame}
%----------- slide --------------------------------------------------%
\begin{frame}
\frametitle{CrIS ILS}

the \cris\ \ils\ for $\fov_i$ can be represented as

\[\int_{\mbox{\tiny FOV}_i} \!\!\!\! w_i(\theta)\, \sinc(2 \pi
                 d(v - v_0 \cos \theta ))\, d\theta \]

% f_{v_0}(v) = 
% \int_{\mbox{\tiny FOV}_i}
% \int_{a_i}^{b_i}

\begin{itemize}
  \item $d$ is max \opd
  \item $v$ is frequency
  \item $v_0$ is reference or channel frequency
  \item $\sinc(x) = sin(x)/x$ for $x \ne 0$,  $1$ for $x = 0$.
  \item $\sinc(2 \pi d(v - v_0 \cos \theta ))$ gives the ILS for a
    single ray at off-axis angle $\theta$
  \item integration is over the intersection of on-axis arcs with
    $\fov_i$, with $w_i(\theta)$ the length of an intersecting arc
    at off-axis angle $\theta$
\end{itemize}

\end{frame}
%----------- slide --------------------------------------------------%
\end{document}

