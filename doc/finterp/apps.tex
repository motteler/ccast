%----------- slide --------------------------------------------------%
\begin{frame}
\frametitle{ASL applications}

\begin{itemize}
  \item finterp.m is a general purpose interpolation routine, now
    part of the ccast package where it is used to interpolate data
    from the sensor to the user grid.  Although it is from 1999, it
    implements all of the ideas above.
  \item kc2cris.m is the latest in a long family of routines with
    the calibration laser frequency or wavelength as a parameter,
    which in turn fixes $dx$.  The intended application is kcarta
    interpolation to the CrIS sensor grid.  kc2cris does not use the
    $dv_1 / dv_2$ ratio to set transform sizes, instead it chooses a
    power of 2 such that $dv_1 <= dvk$, the kcarta step size, and
    interpolates from the $dvk$ to the $dv_1$ grids as a preliminary
    step.
  \item the old fconvkc family of routines.  These are similar to
    kc2cris but may also have a constraint on max transform size.
\end{itemize}
    
For applications such as convolving kcarta radiances to a more
regular user grid (rather than the sensor grid) $dv_1 / dv_2$ will
typically have a tractable rational representation, allowing for 
an exact interpolation.

\end{frame}
